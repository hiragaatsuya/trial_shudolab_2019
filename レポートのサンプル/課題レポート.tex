\documentclass{ujarticle}
\usepackage{geometry}
\geometry{left=20mm,right=20mm,top=20mm,bottom=20mm}

\title{研究プロジェクト課題レポート}
\author{}
\begin{document}
\date{}
\maketitle

%構成の一例です.自由に変えてもらっていいです.

\section{はじめに}

%本レポートの概要を書く.

\section{準備}

%グラフや各種統計量の定義,アクセスモデル,サンプリング手法 (BFS, RW, MHRW, RWRW)の概要,など.
%学んだことは積極的に書いて良いです.

\section{課題: ソーシャルネットワークの次数分布推定}

%以下の2つの内容を含めてください.
%(1)与えられたデータセットにおける,BFS,RW,MHRW, RWRWの4つの手法による次数分布推定の結果と考察.
%(2)ソーシャルネットワークの次数分布をより高精度に推定するための改善案・新しい知見・アイディアなど.

%参考文献 Walking in Facebook: A Case Study of Unbiased Sampling of OSNs [Gjoka et al., 2010]を含めてください.

\section{おわりに}

%感想など.

\bibliography{reference}
\bibliographystyle{junsrt} 

\end{document}